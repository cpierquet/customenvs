% !TeX TXS-program:compile = txs:///arara
% arara: pdflatex: {shell: yes, synctex: no, interaction: batchmode}
% arara: pdflatex: {shell: yes, synctex: no, interaction: batchmode} if found('log', '(undefined references|Please rerun|Rerun to get)')

\documentclass[french,11pt,a4paper]{article}
\usepackage[utf8]{inputenc}
\usepackage[T1]{fontenc}
\RequirePackage[upright]{fourier}
\RequirePackage{mathtools}
\RequirePackage{amsmath,amssymb,amstext}
\RequirePackage[scaled=0.875]{helvet}
\renewcommand\ttdefault{lmtt}
\RequirePackage[scaled=0.925]{cabin} % sf
%\usepackage{DejaVuSerif}
%\usepackage[scale=1.1]{inconsolata}
\usepackage{customenvs}
\usepackage{tabularx}
\usepackage{soul}
%\usepackage{codehigh}
\usepackage{fontawesome5}
\usepackage{lipsum}
\usepackage{fancyvrb}
\usepackage{fancyhdr}
\fancyhf{}
\renewcommand{\headrulewidth}{0pt}
\lfoot{\sffamily\small [customenvs]}
\cfoot{\sffamily\small - \thepage{} -}
\rfoot{\hyperlink{matoc}{\small\faArrowAltCircleUp[regular]}}
\usepackage{hologo}
\providecommand\tikzlogo{Ti\textit{k}Z}
\providecommand\TeXLive{\TeX{}Live\xspace}
\providecommand\PSTricks{\textsf{PSTricks}\xspace}
\let\pstricks\PSTricks
\let\TikZ\tikzlogo

\usepackage{hyperref}
\urlstyle{same}
\hypersetup{pdfborder=0 0 0}
\usepackage[margin=1.5cm]{geometry}
\setlength{\parindent}{0pt}

\def\TPversion{0.1.0}
\def\TPdate{22 octobre 2023}
\usepackage[most]{tcolorbox}
\tcbuselibrary{listingsutf8}
\newtcblisting{DemoCode}[1]{%
	enhanced,width=0.95\linewidth,center,%
	bicolor,size=title,%
	colback=cyan!5!white,%
	colbacklower=cyan!1!white,%
	colframe=cyan!75!black,%
	listing options={%
		breaklines=true,%
		breakatwhitespace=true,%
		style=tcblatex,basicstyle=\small\ttfamily,%
		tabsize=4,%
		commentstyle={\itshape\color{gray}},
		keywordstyle={\color{blue}},%
		classoffset=0,%
		keywords={usepackage,displaystyle,frac,infty,begin,end,lipsum,centering,par,baselineskip,item,bullet,int},%
		alsoletter={-},%
		keywordstyle={\color{blue}},%
		classoffset=1,%
		alsoletter={-},%
		morekeywords={center,justify},%
		keywordstyle={\color{violet}},%
		classoffset=2,%
		alsoletter={-},%
		morekeywords={\ReponsesQCM,MultiCols,\CreerListeItems,\ListeChoixItems},%
		keywordstyle={\color{green!50!black}},%
		classoffset=3,%
		morekeywords={Largeur,Filets,EspacesCL,NbCols,Labels,PoliceLabels,EspaceLabels,Swap,Type,CoeffEspVert,EpTrait,Alea},%
		keywordstyle={\color{orange}}
	},%
	#1
}
\sethlcolor{lightgray!25}
\NewDocumentCommand\MontreCode{ m }{%
	\hl{\vphantom{\texttt{pf}}\texttt{#1}}%
}

\usepackage{babel}

\begin{document}

\pagestyle{fancy}

\thispagestyle{empty}

\begin{center}
	\begin{minipage}{0.75\linewidth}
	\begin{tcolorbox}[colframe=yellow,colback=yellow!15]
		\begin{center}
			\renewcommand\arraystretch{1.25}
			\begin{tabular}{c}
				{\Huge \texttt{customenvs [fr]}}\\
				\\
				{\Large Quelques environnements classiques,} \\
				{\Large légèrement modifiés, et basés} \\
				{\Large sur des environnements existants.} \\
			\end{tabular}
			\renewcommand\arraystretch{1}
			
			\medskip
			
			{\small \texttt{Version \TPversion{} -- \TPdate}}
		\end{center}
	\end{tcolorbox}
\end{minipage}
\end{center}

\vspace*{1mm}

\begin{center}
	\begin{tabular}{c}
	\texttt{Cédric Pierquet}\\
	{\ttfamily c pierquet -- at -- outlook . fr}\\
	\texttt{\url{https://github.com/cpierquet/customenvs}}
\end{tabular}
\end{center}

\vspace*{5mm}

%\hrule
%
%\vspace*{2mm}
%
%{\centering Le nom du package vient de \textsf{ENVironnemenTS ALTernatifs}, avec le suffixe \textsf{FRançais}.\par}
%
%\vspace*{2mm}
%
%\hrule
%
%\vspace*{1cm}

\hrule

\phantomsection

\hypertarget{matoc}{}

\tableofcontents

\vspace*{5mm}

\hrule

\vfill

\section{Historique}

\verb|v0.1.0|~:~~~~Version initiale

\vspace*{5mm}

\pagebreak

\section{Le package customenvs}

\subsection{Idée}

L'idée est de proposer des commandes ou environnements classiques avec quelques éléments de personnalisation (via des \textsf{clés} francisées), comme :

\begin{itemize}
	%\item modifier localement l'\textit{alignement vertical} d'éléments sans jambage ;
	\item \textit{centrer} avec gestion des espacements autour ;
	\item écrire en \textit{multi-colonnes} avec gestion des espacements autour ;
	\item mettre en forme des réponses à des \textit{QCM} ;
	\item créer une liste avec \textit{choix des items} (de manière aléatoire ou par saisie directe).
\end{itemize}

\smallskip

L'idée globale est de proposer des environnements clé en main, avec personnalisations \textit{explicites}, sans forcément avoir besoin de \textit{se pencher sur le code}, mais il est évident qu'il existe d'autres solutions pour l'utilisateur qui souhaite réellement contrôler son rendu.

\smallskip

Il est ici essentiellement question de gérer les espacements, donc on peut citer comme autres solutions possibles :

\begin{itemize}
	\item l'utilisation de \MontreCode{\textbackslash vspace} ou de \MontreCode{\textbackslash  setlength} ;
	\item le package \MontreCode{spacingtricks}.
\end{itemize}

\subsection{Chargement}

Le package se charge dans le préambule, via \MontreCode{\textbackslash usepackage\{customenvs\}}.

Les packages chargés sont :

\begin{itemize}
	\item \MontreCode{xstring}, \MontreCode{simplekv}, \MontreCode{listofitems}, \MontreCode{randomlist} et \MontreCode{xintexpr} ;
	\item \MontreCode{enumitem} ;
	\item \MontreCode{multicol} ;
	\item \MontreCode{tabularray}.
\end{itemize}

À noter que, pour des raisons de compatibilité (ou d'incompatibilité), les packages \MontreCode{enumitem} ou \MontreCode{multicol} ou \MontreCode{tabularray} peuvent ne pas être chargés par \MontreCode{customenvs} (auxquels cas l'utilisateur devra les avoir chargés pour faire fonctionner certains environnements) via les options :

\begin{itemize}
	\item \MontreCode{$\mathtt{\langle}$nonenum$\mathtt{\rangle}$} ;
	\item \MontreCode{$\mathtt{\langle}$nonmulticol$\mathtt{\rangle}$} ;
	\item \MontreCode{$\mathtt{\langle}$nontblr$\mathtt{\rangle}$}.
\end{itemize}

\begin{DemoCode}{listing only}
%chargement avec tous les packages
\usepackage{customenvs}

%chargement avec options(s) pour ne pas charger certains packages
\usepackage[option(s)]{customenvs}
\end{DemoCode}

\newpage

\section{Présentation de réponses à un QCM}

\subsection{Principe}

L'idée est de proposer une environnement prêt à l'emploi pour présenter, grâce à \MontreCode{tabularray} (et non pas à \MontreCode{multicols}) qui est donc à charger, les réponses à une question type QCM, données en colonnes.

\smallskip

Il est possible de spécifier 2, 3 ou 4 réponses, et dans le cas de 4 réponses il est possible de spécifier 1 ou 2 colonnes.

\begin{DemoCode}{listing only}
\ReponsesQCM[options]{liste reponses}<options tblr>
\end{DemoCode}

Les \MontreCode{options} disponibles sont :

\begin{itemize}
	\item \MontreCode{Largeur} pour spécifier la largeur du tableau, \MontreCode{0.99\textbackslash linewidth} par défaut ;
	\item \MontreCode{Filets} pour afficher les filets, \MontreCode{false} par défaut ;
	\item \MontreCode{EspacesCL} pour les espacements Colonnes/Lignes, sous la forme \MontreCode{col/lign} ou \MontreCode{globale}, et valant \MontreCode{6pt/2pt} par défaut ;
	\item \MontreCode{NbCols} pour forcer le passage à 2 colonnes dans le cas de 4 réponses, \MontreCode{4} par défaut ;
	\item \MontreCode{Labels} pour spécifier le formatage des labels, avec \MontreCode{a.} par défaut ;
	\begin{itemize}
		\item pouvant faire intervenir \MontreCode{a} pour \textit{numéroter} \MontreCode{a b c d} ;
		\item pouvant faire intervenir \MontreCode{A} pour \textit{numéroter} \MontreCode{A B C D} ;
		\item pouvant faire intervenir \MontreCode{1} pour \textit{numéroter} \MontreCode{1 2 3 4} ;
	\end{itemize}
	\item \MontreCode{PoliceLabels} pour la police des labels, \MontreCode{\textbackslash bfseries} par défaut ;
	\item \MontreCode{EspaceLabels} pour gérer l'espacement entre le label et la réponse, et valant \MontreCode{\textbackslash kern5pt} par défaut ;
	\item \MontreCode{Swap} pour afficher les (4) réponses en mode 2 colonnes sous la forme ACBD ou ABCD, et valant \MontreCode{false} par défaut.
\end{itemize}

La liste des réponses est à donner sous la forme \MontreCode{answA § answB § ...}

Les options spécifiques, optionnelles et entre \MontreCode{<...>}, sont pour le dernier argument.

\subsection{Exemples}

\begin{DemoCode}{}
%sortie par defaut
\ReponsesQCM{Réponse A § Réponse B § Réponse C § Réponse D}
\end{DemoCode}

\begin{DemoCode}{}
\ReponsesQCM[Filets]{Réponse A § Réponse B § Réponse C § Réponse D}
\end{DemoCode}

\begin{DemoCode}{}
\ReponsesQCM[Filets,Labels=(1.),EspaceLabels={~~~}]{Réponse A § Réponse B § Réponse C}
\end{DemoCode}

\begin{DemoCode}{}
\ReponsesQCM[Labels={A.},PoliceLabels={\color{red}\bfseries}]%
    {Réponse A § Réponse B § Réponse C § Réponse D}
\end{DemoCode}

\begin{DemoCode}{}
\ReponsesQCM[NbCols=4,Labels={A.},PoliceLabels={\color{red}\bfseries}]%
    {Réponse A § Réponse B § Réponse C § Réponse D}
\end{DemoCode}

\begin{DemoCode}{}
\ReponsesQCM[NbCols=2,Labels={A.},PoliceLabels={\color{red}\bfseries}]%
    {Réponse A § Réponse B § Réponse C § Réponse D}
\end{DemoCode}

\begin{DemoCode}{}
\ReponsesQCM[NbCols=2,Swap,Labels={A.},PoliceLabels={\color{red}\bfseries}]%
    {Réponse A § Réponse B § Réponse C § Réponse D}
\end{DemoCode}

\begin{DemoCode}{}
\ReponsesQCM[Filets,NbCols=2,EspacesCL=6pt/10pt]%
    {Réponse A § Réponse B § Réponse C § Réponse D}
\end{DemoCode}

\begin{DemoCode}{}
\ReponsesQCM[Largeur=10cm,NbCols=2,Filets]%
    {$\displaystyle\frac1x$ § $1+\displaystyle\frac1x$ § $-2x^2+5$ § $-\infty$}
    <rows={2cm}>
\end{DemoCode}

\pagebreak

\section{Environnement Centrage}

\subsection{Principe}

L'idée est de proposer un environnement, basé sur \MontreCode{center}, avec une gestion plus fine des espacements avant et après.

Le fait est qu'un environnement \MontreCode{center} génère des espacements (parfois) un peu trop grands autour (on peut également utiliser \MontreCode{\textbackslash centering} ou \MontreCode{\textbackslash centered} du package \MontreCode{spacingtricks}), et donc il s'agit ici de garder l'architecture \textit{environnement} et proposant des solutions pour modifier les espacements.

\begin{DemoCode}{listing only}
\begin{Centrage}[options]
    %corps
\end{Centrage}
\end{DemoCode}

Les \MontreCode{options} disponibles sont :

\begin{itemize}
	\item \MontreCode{Avant} pour spécifier l'espacement avant l'environnement, \MontreCode{0.33\textbackslash baselineskip} par défaut ;
	\item \MontreCode{Apres} pour spécifier l'espacement après l'environnement, \MontreCode{0.33\textbackslash baselineskip} par défaut.
\end{itemize}

À noter que les espacements peuvent être donnés de manière absolue, ou via des dimensions existantes.

\subsection{Exemples}

\begin{DemoCode}{}
%environnement center, par défaut
\lipsum[1][1-3]

\begin{center}
    \lipsum[1][1]
\end{center}

\lipsum[1][1-2]
\end{DemoCode}

\begin{DemoCode}{}
%centering
\lipsum[1][1-3]\par

{\centering\lipsum[1][1]\par}

\lipsum[1][1-2]
\end{DemoCode}

\begin{DemoCode}{}
%environnement Centrage, par défaut
\lipsum[1][1-3]

\begin{Centrage}
    \lipsum[1][1]
\end{Centrage}

\lipsum[1][1-2]
\end{DemoCode}

\begin{DemoCode}{}
%environnement Centrage, personnalisé
\lipsum[1][1-3]

\begin{Centrage}[Avant=0pt,Apres=0pt]
    \lipsum[1][1]
\end{Centrage}

\lipsum[1][1-2]
\end{DemoCode}

\begin{DemoCode}{}
%environnement Centrage, avec des listes
\lipsum[2][3]

\begin{itemize}
    \item \lipsum[1][1]
    \item \lipsum[1][2]
\end{itemize}

\begin{Centrage}[Avant=-0.25\baselineskip]
    \lipsum[1][1]
\end{Centrage}

\lipsum[1][1-2]
\end{DemoCode}

\pagebreak

\section{Environnement multi-colonnes}

\subsection{Principe}

L'idée est de proposer un environnement basé sur \MontreCode{multicols} (donc le package \MontreCode{multicol} est à charger), pour lequel les espacements avant et après peuvent être personnalisés.

C'est la longueur \MontreCode{\textbackslash multicolsep}, qui vaut \MontreCode{12pt plus 4pt minus 3pt} par défaut, qui gère ces espacements.

\smallskip

L'idée est donc de proposer une environnement \textit{simplifié} intégrant une modification de cette longueur. Il sera également possible de créer automatiquement un environnement multi-colonnes combiné avec une liste d'énumération (avec \MontreCode{enumitem} chargé, par exemple) !

De plus, si le multi-colonnes est destiné à accueillir une liste, les items seront correctement alignés avec une liste sans multi-colonnes.

\begin{DemoCode}{listing only}
\begin{MultiCols}[options](nbcols)<options enumitem>
    %corps
\end{MutiCols}
\end{DemoCode}

Les \MontreCode{options} disponibles sont :

\begin{itemize}
	\item \MontreCode{Type} pour spécifier le type d'environnement qui sera inclus en multi-colonnes, et valant \MontreCode{texte} par défaut ;
	
	\hfill{}à choisir parmi \MontreCode{texte / enum / item}
	\item \MontreCode{CoeffEspVert} pour spécifier le coefficient à appliquer à la longueur par défaut, et valant \MontreCode{0.5} par défaut ;
	
	\hfill{}à choisir parmi \MontreCode{0 / 0.25 / 0.33 / 0.5 / 0.66 / 0.75 / 1 / 1.25}
	\item \MontreCode{EpTrait} pour l'épaisseur éventuelle du trait de séparation, et valant \MontreCode{0pt} par défaut.
\end{itemize}

Le nombre de colonnes, obligatoire, est à donner entre \MontreCode{(...)}.

L'argument optionnel et entre \MontreCode{<...>} est passé à l'environnement \MontreCode{enumitem} ou \MontreCode{itemize} si spécifié.

\subsection{Exemples}

\begin{DemoCode}{}
%par défaut
\lipsum[1][1-2]

\begin{MultiCols}(2)
    \lipsum[2]
\end{MultiCols}

\lipsum[1][3-4]
\end{DemoCode}

\begin{DemoCode}{}
%espacement réduit + filet
\lipsum[1][1-2]

\begin{MultiCols}[CoeffEspVert=0.25,EpTrait=1pt](3)
    \lipsum[2]
\end{MultiCols}

\lipsum[1][3-4]
\end{DemoCode}

\begin{DemoCode}{}
%type enumitem
\begin{enumerate}
    \item \lipsum[1][1-2]
    \begin{MultiCols}[Type=enum](4)
        \item bla
        \item bla
        \item bla
        \item bla
    \end{MultiCols}
    \item \lipsum[1][3-4]
    \begin{MultiCols}[Type=item](3)<label=$\bullet$>
        \item bla
        \item bla
        \item bla
    \end{MultiCols}
\end{enumerate}

\lipsum[3][1]
\end{DemoCode}

\pagebreak

\section{Énumération avec choix des items, parmi une liste}

\subsection{Principe et fonctionnement}

L'idée est ici de :

\begin{itemize}
	\item créer une liste d'items qui servira de base pour le(s) choix ;
	\item afficher la liste avec choix des items, de manière aléatoire ou par items choisis
\end{itemize}

À noter que l'environnement \MontreCode{MultiCols} du package peut être utilisé comme environnement de listes !

\begin{DemoCode}{listing only}
\CreerListeItems{liste}{macro}{nomliste}
\end{DemoCode}

\begin{DemoCode}{listing only}
\ListeChoixItems[clés]{macro}{nomliste}(numéros)<options enumitem>
\end{DemoCode}

Les \MontreCode{clés} disponibles sont :

\begin{itemize}
	\item \MontreCode{Type} pour spécifier le type d'environnement, et valant \MontreCode{enum} par défaut ;
	
	\hfill{}à choisir parmi \MontreCode{enum} \MontreCode{item} ou \MontreCode{MultiCols/Type/NbCols}
	\item \MontreCode{Alea} pour forcer un affichage aléatoire, \MontreCode{false} par défaut.
\end{itemize}

Le deuxième argument, obligatoire et entre \MontreCode{\{...\}} est la macro créée précédemment.

Le troisième argument, obligatoire et entre \MontreCode{\{...\}} est le nom de la liste créée précédemment.

Le quatrième argument, obligatoire et entre \MontreCode{(...)} permet de spécifier :

\begin{itemize}
	\item le nombre d'items à afficher en mode \MontreCode{Alea=true} ;
	\item les items à afficher, sous la forme \MontreCode{num1,num2,...}.
\end{itemize}

Le dernier argument, optionnel et entre \MontreCode{<...>} correspond à des options spécifiques à passer à l'environnement de liste \MontreCode{enumitem} créé.

\medskip

À noter que des contrôles sont effectués lors de l'appel aux macros pour :

\begin{itemize}
	\item vérifier que la liste n'existe pas déjà (pour la macro de création) ;
	\item vérifier que la liste existe déjà (pour la macro d'affichage des items).
\end{itemize}
\subsection{Exemples}

\begin{DemoCode}{listing only}
%création de la liste ListeItems, avec la macro \malisteditems
\CreerListeItems%
    {Réponse A,Réponse B,Réponse C,Réponse D,Réponse E,Réponse F,Réponse G,Réponse H}%
    {\malisteditems}{ListeItems}
\end{DemoCode}

\CreerListeItems{Réponse A,Réponse B,Réponse C,Réponse D,Réponse E,Réponse F,Réponse G,Réponse H}{\malisteditems}{ListeItems}

\begin{DemoCode}{}
%affichage d'items aléatoires
\ListeChoixItems[Alea]{\malisteditems}{ListeItems}(5)
\end{DemoCode}

\begin{DemoCode}{}
%affichage de certains items
\ListeChoixItems{\malisteditems}{ListeItems}(1,4,3,8,2)
\end{DemoCode}

\begin{DemoCode}{listing only}
%création de la liste ListeItemsB, avec la macro \malisteditemsb
\CreerListeItems%
    {{$\int_0^1 x^2 dx$},{$\int_0^1 x^3 dx$},{$\int_0^1 x^4 dx$},...}%
    {\malisteditemsb}{ListeItemsB}
\end{DemoCode}

\CreerListeItems{{$\int_0^1 x^2 dx$},{$\int_0^1 x^3 dx$},{$\int_0^1 x^4 dx$},{$\int_0^1 x^5 dx$},{$\int_0^1 x^6 dx$},{$\int_0^1 x^7 dx$},{$\int_0^1 x^8 dx$}}{\malisteditemsb}{ListeItemsB}

\begin{DemoCode}{}
%affichage d'items aléatoires, via MultiCols
\ListeChoixItems[Alea,Type={MultiCols/enum/2}]{\malisteditemsb}{ListeItemsB}(4)
\end{DemoCode}

\begin{DemoCode}{}
%affichage de certains items
\ListeChoixItems[Type=item]{\malisteditemsb}{ListeItemsB}(7,2,1,5,3)<label=$\bullet$>
\end{DemoCode}

\end{document}